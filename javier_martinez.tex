\documentclass{simplecv}
\usepackage[utf8]{inputenc}
%\usepackage[latin1]{inputenc}
\usepackage[spanish]{babel}
\usepackage{hyperref}
\usepackage{url}

\begin{document}

\leftheader{+595 981 886 658\\
\texttt{javier@dowhile0.org}\\
\texttt{Veteranos de la guerra del 70, 581, Asuncion, Paraguay}}

\title{Javier Martinez Canillas}

\maketitle

\section{Summary}

I'm a software engineer with experience working on different layers of the Linux software stack,
ranging from the kernel to user-space applications.\\

Also, I'm an open-source enthusiast that contributes to many open source projects specially the
Linux kernel, where I've contributed 800+ patches and maintain a couple of components. \\

\url{https://www.openhub.net/accounts/martinezjavier}

\section{Specialties}

Software engineering, Linux user-space and kernel development, System integration, Packaging.

\section{Education}

\begin{topic}

\item[2010 - 2011] {\bf Master of Science (M.Sc.), High Performance Computing} - Universitat Autónoma de Barcelona, Bellaterra, Spain. 

{\bf Thesis:} \emph{Including the Workload Effect in the Parallel Program Signature}.

Advisor: Prof. Emilio Luque.

\item[2002 - 2008] {\bf Bachelor of Science (B.Sc.), Informatics Engineering} - Catholic University "Nuestra Señora de la Asunción", Asunción, Paraguay. 

{\bf Thesis:} \emph{A new approach to solve Multi-Objective Evolutionary Optimization Problems using Linear Genetic Programming}. 

Advisor: Prof. Benjamín Barán.

\end{topic}

\section{Awards and Honors}

PIF-UAB predoctoral scholarship for Master of Science studies at Universitat Autónoma de Barcelona.

\section{Professional Experience}

\begin{topic}

\item[July 2015 - Present] \emph{Senior Linux Kernel Developer} - Samsung Research America (\url{http://www.sra.samsung.com/})

\begin{itemize}

\item Member of the Open Source Group working to improve upstream FOSS projects, specially the Linux kernel.

\item Helping product teams to extend FOSS components to fit their needs.

\end{itemize}

\item[January 2012 - June 2015] \emph{Senior Software Engineer} - Collabora (\url{http://www.collabora.co.uk})

\begin{itemize}

\item Helped customers to upstream the Linux kernel support for their platforms, to work close to mainline and reduce their maintenance burden.

\item Developed customer specific Linux distributions using the OpenEmbedded/BitBake build system and Debian/Ubuntu derivatives using Debian packaging and the Open Build Service (OBS).

\item Improved the Desktop Bus (D-Bus) inter-process communication system performance by developing a new socket address family (AF\_BUS) for multicast communications on Linux(\url{http://projects.genivi.org/afbus-dbus-optimization/contact-us}).

\end{itemize}

Repository: \url{http://cgit.collabora.com/git/user/javier}

\item[September 2011 - January 2012] \emph{Linux Kernel Engineer} - ISEE (\url{http://www.isee.biz}): Embedded systems manufacturer.

\begin{itemize}

\item Linux kernel device drivers and bootloader (X-loader and U-boot) development for custom ARM OMAP3 (Cortex-A8) based boards.

\end{itemize}

Repository: \url{http://git.igep.es/}

\item[October 2010 - August 2011] \emph{M.Sc. Student, Professor and Researcher} - Computer Architecture and Operating System department, Universitat Autónoma de Barcelona.

\begin{itemize}

\item Design and implementation of a parallel application performance analysis and prediction tool.

\item Doing research to improve the tool accuracy.

\item Teaching assistant for the course ``Electronic systems design based on microcontrollers''.

\end{itemize}

\item[July 2009 - September 2010] Professor and Researcher  - Computer Engineering, Polytechnic Faculty, National University of Asunción.

\begin{itemize}

\item Updated the Linux Device Drivers 3 book examples to compile and run with newer kernels. So students can develop their own virtual device drivers based in those examples:

Repository: \url{https://github.com/martinezjavier/ldd3}

\end{itemize}

\end{topic}

\section{Open source contributions}

\begin{topic}

\item[May 2010 - Present] \emph{Kernel hacker} - Linux kernel

Mainline maintainer of the OMAP3 IGEP embedded board family and the Maxim MAX77802 PMIC drivers.

\url{https://git.kernel.org/cgit/linux/kernel/git/torvalds/linux.git/log/?qt=author&q=Javier+Martinez}

\item[Jun 2016 - Present] \emph{Author and maintainer} - RoDI robot ROS package

Robot Operating System package for the RoDi robot

\url{https://github.com/rodibot/rodi_robot}

\end{topic}

\section{Publications in Journals and Conferences}

\begin{thebibliography}{10}

\footnotesize

\bibitem{mar15}
Javier Martinez Canillas. \emph{A Survivor's Guide to Contributing to the Linux Kernel}, Korea Linux Forum, 2015.

\bibitem{mar12a}
Javier Martinez Canillas. \emph{Kbuild: the Linux Kernel Build System}, The Linux Journal, 2012. 

\bibitem{mar11b}
Martinez Canillas, J. and Wong, A. and Rexachs, D. and Luque, E. \emph{Including the Workload Effect in the Parallel Program Signature}. High Performance Computing and Communications (HPCC), 2011 IEEE International Conference on. September, 2011, Banff, Canada.

\bibitem{mar11a}
Martinez Canillas, J. and Wong, A. and Rexachs, D. and Luque, E. \emph{Predicting Parallel Applications Performance using Signatures: the Workload Effect}. Computer Systems and Applications (AICCSA), 2011 IEEE/ACS International Conference on. December, 2011, Sharm El-Sheikh, Egypt.

\bibitem{mar08b}
R. Sánchez, J. Martinez, B. Barán. \emph{Macro-Economic Time-Series Forecasting Using Linear Genetic Programming}, $11^{th}$ Joint Conference on Information Sciences, Dec 15-20, 2008, China. 

\bibitem{mar08a}
Javier Martinez Canillas, Roberto Sánchez, Benjamín Barán. \emph{Estimation Models Generation using Linear Genetic Programming}. CLEI Electronic Journal Volume 12 Number 3, December 2009. Regular Issue and Special Issue of Best Papers presented at CLEI 2008, Santa Fe, Argentina.

\end{thebibliography}

\end{document}
